%%%%%%%%%%%%%%%%%%%%%%%%%%%%%%%%%%%%%%%%%%%%%%%%%%%%%%%%%%%%%%%%%%%%%%%%%%%%%%%%
%2345678901234567890123456789012345678901234567890123456789012345678901234567890
%        1         2         3         4         5         6         7         8

\documentclass[a4, 10 pt, conference]{ieeeconf}  % Comment this line out if you need a4paper

%\documentclass[a4paper, 10pt, conference]{ieeeconf}      % Use this line for a4 paper

\IEEEoverridecommandlockouts                              % This command is only needed if 
                                                          % you want to use the \thanks command

\overrideIEEEmargins                                      % Needed to meet printer requirements.

% See the \addtolength command later in the file to balance the column lengths
% on the last page of the document

% The following packages can be found on http:\\www.ctan.org
%\usepackage{graphics} % for pdf, bitmapped graphics files
%\usepackage{epsfig} % for postscript graphics files
%\usepackage{mathptmx} % assumes new font selection scheme installed
%\usepackage{times} % assumes new font selection scheme installed
%\usepackage{amsmath} % assumes amsmath package installed
%\usepackage{amssymb}  % assumes amsmath package installed
\usepackage{multicol}
\usepackage{tcolorbox}
\usepackage{cuted,tcolorbox,lipsum}
\usepackage{xcolor}

\title{\LARGE \bf
Introduction to Machine Learning (SS 2023)\\ Programming Project
\vspace{-3em}
}


%\author{Someone Anyone$^{1}$ and Xiang Zhang$^{2}$% <-this % stops a space
%}


\begin{document}


\maketitle
\vspace{-3em}
\thispagestyle{empty}
\pagestyle{empty}

\begin{strip}
\begin{tcolorbox}[
size=tight,
colback=white,
boxrule=0.2mm,
left=3mm,right=3mm, top=3mm, bottom=1mm
]
{\begin{multicols}{3}% replace 3 with 2 for 2 authors.

\textbf{Author 1}       \\
Last name:              \\  % Enter first name
First name:             \\  % Enter first name
Matrikel Nr.:               \\  % Enter Matrikel number

\columnbreak

\textbf{Author 2}       \\ 
Last name:  Untermann            \\  % Enter first name
First name:   Sven          \\  % Enter first name
Matrikel Nr.:   11906555           \\  % Enter Matrikel number

\columnbreak

% only four three person team
% \textbf{Author 3}       \\
% Last name:              \\  % Enter first name
% First name:             \\  % Enter first name
% Matrikel Nr.:               \\  % Enter Matrikel number

\end{multicols}}
\end{tcolorbox}
\end{strip}

%%%%%%%%%%%%%%%%%%%%%%%%%%%%%%%%%%%%%%%%%%%%%%%%%%%%%%%%%%%%%%%%%%%%%%%%%%%%%%%%


%{\color{blue}
 % \noindent This template outlines the sections that your report must 
  %contain. Inside each section, we provide pointers to what you should
 % write about in that section (in blue text).  \linebreak

%\noindent \textbf{Please remove all the text in blue in your report!
 % Your report should be 2 pages for regular teams (excluding references!)
  %and 3 pages for the three person team.}  }

\section{Introduction}
\label{sec:intro}

{\color{blue}

\begin{itemize}
	The ability to identify fraudulent transactions is of great interest to the payments industry. In this project, we will utilize a binary classifier trained on a transactions dataset to detect fraud. The dataset consists of various features related to the transactions, along with the amount of each transaction. Our objective is to determine, based on these features, whether a transaction is fraudulent or not. The dataset contains 227,845 instances of transactions with 29 different features. Throughout this project, we will train and evaluate the classifier using these data
\end{itemize}
}


\section{Implementation / ML Process}
\label{sec:methods}

{\color{blue}

\begin{itemize}
	\item The given dataset consists of transaction data with a binary classification task of identifying fraud transactions. Before selecting a model, it is important to analyze the data and address any issues, such as class imbalance.

	\item The dataset is highly imbalanced, with a majority of legitimate transactions (Class 0) and a minority of fraud transactions (Class 1). To balance the dataset, the "Class 1" samples can be oversampled using the resample function from scikit-learn. This creates a new dataset with an equal number of samples for both classes. The balanced dataset is then used for training the models.

	\item After balancing the dataset, we can proceed with selecting a suitable model. The script provides options for different algorithms, including Logistic Regression, Decision Trees, Random Forests, Multilayer Perceptron (Neural Network), and AdaBoost. Each algorithm is defined with its corresponding hyperparameter search space.

	\item The chosen algorithm family for this problem is ensemble methods, which includes Decision Trees, Random Forests, and AdaBoost. Ensemble methods combine multiple weak models to create a stronger model with improved generalization and robustness. These algorithms are suitable for this problem because they can handle complex relationships in the data and have the ability to learn from imbalanced datasets.

	\item The hyperparameters for each algorithm are chosen using grid search with cross-validation.\\ \\[1em] This approach 
	systematically tests different combinations of hyperparameters and selects the best performing set. The evaluation metric used for hyperparameter selection is ROC AUC (Receiver Operating Characteristic Area Under the Curve), which is a common metric for imbalanced classification problems.

	\item For the final model, the best hyperparameters obtained from grid search are used. The hyperparameters are as follows:

\hspace*{0.5cm}Logistic Regression:\newline
\hspace*{1cm}C = 0.1,\\ \hspace*{1cm}class\_weight = \{0: 0.1, 1: 0.9\}\newline
\hspace*{0.5cm}Decision Tree:\newline
\hspace*{1cm}max\_depth = 15,\newline
\hspace*{1cm}class\_weight = 'balanced'\newline
\hspace*{0.5cm}Random Forest:\newline
\hspace*{1cm}n\_estimators = 100,\newline
\hspace*{1cm}max\_features = 'sqrt',\newline
\hspace*{1cm}class\_weight = 'balanced'\newline
\hspace*{0.5cm}Multilayer Perceptron:\newline
\hspace*{1cm}hidden\_layer\_sizes = (50, 50),\newline
\hspace*{1cm}activation = 'relu',\newline
\hspace*{1cm}alpha = 0.0001\newline
\hspace*{0.5cm}AdaBoost:\newline
\hspace*{1cm}n\_estimators = 100,\newline
\hspace*{1cm}estimator =DecisionTreeClassifier(max\_depth=4)

	\item These hyperparameters are chosen based on their performance in terms of ROC AUC and accuracy scores on the testing data. The final models are trained using these hyperparameters.

	\item It is important to note that the provided script does not cover the entire data preprocessing and feature engineering pipeline. It assumes that the data is already preprocessed and ready for model training. In a real-world scenario, additional steps such as data cleaning, feature scaling, and feature selection may be required before training the models.
\end{itemize}
}

\section{Results}
\label{sec:results}

{\color{blue}

\begin{itemize}
	\item Describe the performance of your model (in terms of the metrics for your dataset) on the training and validation sets with the help of plots or/and tables.
	\item You must provide at least two separate visualizations
          (plot or tables) of different things, i.e. don’t use a table
          and a bar plot of the same metrics. At least three
           visualizations are required for the 3 person team.
\end{itemize}
}

\section{Discussion}
\label{sec:discuss}

{\color{blue}
\begin{itemize}
	\item Analyze the results presented in the report (comment on what contributed to the good or bad results). If your method does not work well, try to analyze why this is the case.
	\item Describe very briefly what you tried but did not keep for your final implementation (e.g. things you tried but that did not work, discarded ideas, etc.).
	\item How could you try to improve your results? What else would you want to try?

\end{itemize}
}

\section{Conclusion}
\label{sec:con}

{\color{blue}

  \begin{itemize}
  \item Finally, describe the test-set performance you achieved. Do not
    optimize your method based on the test set performance!
  \item Write a 5-10 line paragraph describing the main takeaway of your project.
  \end{itemize}

}

%%%%%%%%%%%%%%%%%%%%%%%%%%%%%%%%%%%%%%%%%%%%%%%%%%%%%%%%%%%%%%%%%%%%%%%%%%%%%%%%



\end{document}
